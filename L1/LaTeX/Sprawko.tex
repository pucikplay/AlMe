% uklad dokumentu
	\documentclass{article}
	\usepackage{xparse}
	\usepackage[margin=1.5cm]{geometry}
    \usepackage{enumerate} 
	\frenchspacing
    \linespread{1.0}
    \setlength{\parindent}{0pt}

% jezyk polski
	\usepackage[T1]{fontenc}
	\usepackage[polish]{babel}
	\usepackage[utf8]{inputenc}
 
% pakiety matematyczne
    \let\lll\undefined
    \usepackage{mathtools}
	\usepackage{amssymb}
    \usepackage{amsthm}
	\usepackage{amsmath}
	\usepackage{amsfonts}
	\usepackage{tikz}
	
% pakiety do automatów
	\usetikzlibrary{automata, arrows.meta, positioning, arrows}

    \title{\textbf{Algorytmy Metaheurystyczne\\Komiwojażer Heurystycznie}}
    \author{Gabriel Budziński(254609)\\Franciszek Stepek (256310)}
    \date{}
    
    
\begin{document} 
\maketitle

\section*{Przedmowa}
Sprawko na metaheurystykę

\section{Podsumowanie złożoności obliczeniowych implementacji}
\subsection{K-Random}
Co nieco o nim

\subsection{Nearest Neighbor}
Co nieco o nim

\subsection{Nearest Branching Neighbor}
Co nieco o nim

\subsection{2-Opt}
Co nieco o nim

\subsection{3-Opt}
Co nieco o nim

\section{Opis eksperymentów}
\subsection{Implementacja}
C/Cpp

\subsection{Sprzęt}
16GB Ramu

\subsubsection{Pececik}
6 rdzonków
\subsubsection{Lapek}
4 rdzonki

\subsection{Instancje}
\subsubsection{Przykładziki ze stronki}

\subsubsection{Instancje losowe}

\subsection{Metodologia/cel}

\subsection{Opis wyników}
Testy Wilcoxona dla poszczególnych algorytmów:
\begin{itemize}
	\item W pierwszej dużej tabeli będziemy mieli zestawienie 3 algorytmów
	\item W drugiej i trzeciej (Już nieco 'ciekawszej') Zobaczymy jak na rozwiązanie \textit{2-Opt} wpływają warunki początkowe (zarówno czasowo, jak i w kontekście funkcji Celu)
	\item W czwartej i piątej otrzymamy większe zestawienie różnych przypadków dla \textit{Nearest Neighbor}
\end{itemize}

\newpage
\subsubsection{Część I}
Tabela numer 1:
\begin{table}[h!]
\centering
\begin{tabular}{c||c|c|c|c|c c c|c|c} 
Nr. Instancji & 1 & 2 & 3 & 4 & 5 &  & 1-2 & 1-3 & 4-5 \\
\hline
1 & 3157 & 32621 & 2846 & 2922 & 33913 &  & -29464 & 311 & -30991 \\
2 & 8980 & 29990 & 8385 & 8746 & 25809 &  & -21010 & 595 & -17063 \\
3 & 135737 & 610252 & 126892 & 128097 & 622778 &  & -474515 & 8845 & -494681 \\
4 & 7579 & 47194 & 7098 & 6741 & 49975 &  & -39615 & 481 & -43234 \\
5 & 8191 & 53545 & 7157 & 7372 & 54720 &  & -45354 & 1034 & -47348 \\
6 & 803 & 3509 & 714 & 677 & 3560 &  & -2706 & 89 & -2883 \\
7 & 511 & 1586 & 454 & 443 & 1750 &  & -1075 & 57 & -1307 \\
8 & 642 & 2606 & 561 & 595 & 2645 &  & -1964 & 81 & -2050 \\
9 & 27807 & 168501 & 23737 & 23006 & 172442 &  & -140694 & 4070 & -149436 \\
10 & 33633 & 270344 & 28612 & 29756 & 242459 &  & -236711 & 5021 & -212703 \\
11 & 35859 & 331888 & 31749 & 34109 & 345599 &  & -296029 & 4110 & -311490 \\
12 & 29158 & 154024 & 25393 & 25014 & 174054 &  & -124866 & 3765 & -149040 \\
13 & 34499 & 256477 & 28435 & 29312 & 253834 &  & -221978 & 6064 & -224522 \\
14 & 36980 & 338421 & 33375 & 34698 & 339346 &  & -301441 & 3605 & -304648 \\
15 & 26227 & 174278 & 23107 & 22850 & 178774 &  & -148051 & 3120 & -155924 \\
16 & 26947 & 173366 & 24721 & 25011 & 174970 &  & -146419 & 2226 & -149959 \\
17 & 27460 & 170965 & 24459 & 23810 & 154305 &  & -143505 & 3001 & -130495 \\
18 & 20356 & 126612 & 18281 & 15972 & 121878 &  & -106256 & 2075 & -105906 \\
19 & 54019 & 593697 & 49141 & 46246 & 606262 &  & -539678 & 4878 & -560016 \\
20 & 54019 & 582949 & 49141 & 45758 & 575474 &  & -528930 & 4878 & -529716 \\
21 & 68964 & 1405305 & 60930 & 62216 & 1415877 &  & -1336341 & 8034 & -1353661 \\
22 & 331103 & 6434381 & 282758 & 279968 & 6392975 &  & -6103278 & 48345 & -6113007 \\
23 & 46680 & 597495 & 45340 & 49462 & 572145 &  & -550815 & 1340 & -522683 \\
24 & 69297 & 735352 & 61910 & 64640 & 677100 &  & -666055 & 7387 & -612460 \\
25 & 120769 & 842650 & 110103 & 107601 & 882692 &  & -721881 & 10666 & -775091 \\
26 & 61652 & 821860 & 61400 & 68116 & 823348 &  & -760208 & 252 & -755232 \\
27 & 85699 & 988886 & 79996 & 81923 & 1028352 &  & -903187 & 5703 & -946429 \\
28 & 94683 & 1638917 & 87542 & 88659 & 1635424 &  & -1544234 & 7141 & -1546765 \\
29 & 58023 & 1137733 & 56765 & 55439 & 1183487 &  & -1079710 & 1258 & -1128048 \\
30 & 59890 & 766136 & 52403 & 56468 & 791951 &  & -706246 & 7487 & -735483 \\
31 & 131281 & 1940053 & 118468 & 119774 & 1962350 &  & -1808772 & 12813 & -1842576 \\
32 & 153462 & 586378 & 121315 & 121297 & 568936 &  & -432916 & 32147 & -447639 \\
33 & 2752 & 22583 & 2497 & 2684 & 22420 &  & -19831 & 255 & -19736 \\
34 & 8605 & 110285 & 7566 & 7628 & 114431 &  & -101680 & 1039 & -106803 \\
35 & 11054 & 183204 & 9415 & 9627 & 182635 &  & -172150 & 1639 & -173008 \\
36 & 1554 & 8755 & 1418 & 1346 & 8903 &  & -7201 & 136 & -7557 \\
37 & 830 & 3585 & 770 & 775 & 3466 &  & -2755 & 60 & -2691 \\
38 & 152493 & 1629073 & 139697 & 140164 & 1576951 &  & -1476580 & 12796 & -1436787 \\
39 & 5030 & 41053 & 4418 & 4294 & 42391 &  & -36023 & 612 & -38097 \\
 \end{tabular}
\end{table}

Przy czym kolejne kolumny 'numeryczne' oznaczają:
\begin{itemize}
	\item 1 - Wartość funkcji Celu dla algorytmu \textit{Nearest Neighbor} startującego z wybranego 'miasta'
	\item 2 - Wartość funkcji Celu dla algorytmu \textit{K-Random} działającego tak długo, jak z kolumny 1
	\item 3 - Wartość funkcji Celu dla algorytmu \textit{2-Opt} startującego z rozwiązania w punkcie 1
	\item 4 - Wartość funkcji Celu dla algorytmu \textit{2-Opt} startującego z losowej instancji
	\item 5 - Wartość funkcji Celu dla algorytmu \textit{K-Random} działającego tak długo, jak z kolumny 4
\end{itemize}

A teraz jak rozumieć kolejne kolumny 'różnicowe':
\begin{itemize}
	\item 1-2 - Porównanie działania KR z NN dla tego samego budżetu obliczeniowego
	\item 1-3 - 'Dowód', że 2O rzeczywiście poprawia rozwiązanie startowe
	\item 4-5 - Porównanie działania KR z 2O dla tego samego budżetu obliczeniowego
\end{itemize}

Zauważmy, że we wszystkich tych kolumnach każda z wartości jest tego samego znaku, zatem bez dokładniejszej analizy można powiedzieć, że wartość statystyki testowej dla testu Wilcoxona będzie równa 0, zatem jednoznacznie powie nam, który z algorytmów zwraca zawsze lepsze rozwiązanie. Pojawia się nam zatem następująca zależność (gdzie znak '<' oznacza, że lewa wartość zwraca nam 'gorsze' rozwiązanie od prawego):
\[\textit{K-Random} < \textit{Nearest Neighbor} < \textit{2-Opt}\]

\newpage
\subsubsection{Część II}
Tabela numer 2:
\begin{table}[h!]
\centering
\begin{tabular}{c||c|c||c|c||c}
Nr. Instancji & NN & KR & NN-KR & |NN-KR| & Rangi \\
\hline
1 & 2846 & 2992 & -146 & 146 & 11 \\
2 & 8385 & 7967 & 418 & 418 & 16 \\
3 & 126892 & 124006 & 2886 & 2886 & 29 \\
4 & 7098 & 6988 & 110 & 110 & 8 \\
5 & 7157 & 7080 & 77 & 77 & 6.5 \\
6 & 714 & 689 & 25 & 25 & 4 \\
7 & 454 & 461 & -7 & 7 & 2 \\
8 & 561 & 567 & -6 & 6 & 1 \\
9 & 23737 & 24667 & -930 & 930 & 19 \\
10 & 28612 & 29780 & -1168 & 1168 & 21 \\
11 & 31749 & 32634 & -885 & 885 & 18 \\
12 & 25393 & 24066 & 1327 & 1327 & 22 \\
13 & 28435 & 29226 & -791 & 791 & 17 \\
14 & 33375 & 33294 & 81 & 81 & 7 \\
15 & 23107 & 22870 & 237 & 237 & 13 \\
16 & 24721 & 23678 & 1043 & 1043 & 20 \\
17 & 24459 & 24275 & 184 & 184 & 12 \\
18 & 18281 & 15962 & 2319 & 2319 & 26 \\
19 & 49141 & 46422 & 2719 & 2719 & 28 \\
20 & 49141 & 45633 & 3508 & 3508 & 32 \\
21 & 60930 & 61190 & -260 & 260 & 14 \\
22 & 282758 & 287980 & -5222 & 5222 & 34 \\
23 & 45340 & 52126 & -6786 & 6786 & 38 \\
24 & 61910 & 67957 & -6047 & 6047 & 36 \\
25 & 110103 & 103341 & 6762 & 6762 & 37 \\
26 & 61400 & 66836 & -5436 & 5436 & 35 \\
27 & 79996 & 82941 & -2945 & 2945 & 30 \\
28 & 87542 & 90788 & -3246 & 3246 & 31 \\
29 & 56765 & 58313 & -1548 & 1548 & 24 \\
30 & 52403 & 56508 & -4105 & 4105 & 33 \\
31 & 118468 & 119808 & -1340 & 1340 & 23 \\
32 & 121315 & 118891 & 2424 & 2424 & 27 \\
33 & 2497 & 2624 & -127 & 127 & 10 \\
34 & 7566 & 7643 & -77 & 77 & 6.5 \\
35 & 9415 & 9687 & -272 & 272 & 15 \\
36 & 1418 & 1399 & 19 & 19 & 3 \\
37 & 770 & 799 & -29 & 29 & 5 \\
38 & 139697 & 137507 & 2190 & 2190 & 25 \\
39 & 4418 & 4531 & -113 & 113 & 9 \\
\end{tabular}
\end{table}

Przy czym kolejne kolumny oznaczają:
\begin{itemize}
	\item NN - Wartość funkcji Celu dla algorytmu \textit{2-Opt} startującego z rozwiązania znalezionego przez algorytm \textit{Nearest Neighbor}
	\item KR - Wartość funkcji Celu dla algorytmu \textit{2-Opt} startującego z rozwiązania znalezionego przez algorytm \textit{K-Random} (Startowe rozwiązanie w tym samym czasie, co startowe dla NN)
	\item Kolejne 2  kolumny znaczą dokładnie to, co ich nazwa, natomist ostatnia wyznacza rangi dla testu Wilcoxona
\end{itemize}

A teraz policzmy jeszcze tylko 2 sumy dla Wilcoxona:
\begin{itemize}
	\item $T_+ = 315.5$
	\item $T_- = 432.5$
\end{itemize}
Zatem nasza statystyka końcowa będzie wynosiła $315.5$. Trzeba tu teraz coś konstruktywnie dopowiedzieć, dlaczego to oznacza, że nie ma w sumie za bardzo znaczenia z czego startuje...\\
\\
Tabela numer 3:
\begin{table}[h!]
\centering
\begin{tabular}{c||c|c||c}
Nr. Instancji & NN & KR & NN-KR \\
\hline
1 & 0.115728 & 3.10199 & -2.986262 \\
2 & 0.00120772 & 0.00922684 & -0.00801912 \\
3 & 0.0187193 & 0.145382 & -0.1266627 \\
4 & 0.0138299 & 0.184798 & -0.1709681 \\
5 & 0.0393003 & 0.338091 & -0.2987907 \\
6 & 0.00987954 & 0.0552343 & -0.04535476 \\
7 & 0.00119699 & 0.0103625 & -0.00916551 \\
8 & 0.00364293 & 0.0333502 & -0.02970727 \\
9 & 0.0154935 & 0.137051 & -0.1215575 \\
10 & 0.053393 & 0.313875 & -0.260482 \\
11 & 0.144667 & 0.836886 & -0.692219 \\
12 & 0.0138314 & 0.0914137 & -0.0775823 \\
13 & 0.037316 & 0.41158 & -0.374264 \\
14 & 0.109228 & 0.997454 & -0.888226 \\
15 & 0.0119046 & 0.102974 & -0.0910694 \\
16 & 0.01084 & 0.0888192 & -0.0779792 \\
17 & 0.0122842 & 0.114743 & -0.1024588 \\
18 & 0.0165922 & 0.0705111 & -0.0539189 \\
19 & 0.322236 & 4.21766 & -3.895424 \\
20 & 0.343797 & 2.56742 & -2.223623 \\
21 & 27.5015 & 656.948 & -629.4465 \\
22 & 11.1163 & 179.428 & -168.3117 \\
23 & 0.0065809 & 0.110337 & -0.1037561 \\
24 & 0.0144687 & 0.196237 & -0.1817683 \\
25 & 0.0155958 & 0.195165 & -0.1795692 \\
26 & 0.00640445 & 0.246059 & -0.23965455 \\
27 & 0.0186396 & 0.358017 & -0.3393774 \\
28 & 0.062336 & 0.966489 & -0.904153 \\
29 & 0.0857857 & 2.09062 & -2.0048343 \\
30 & 0.221278 & 4.1955 & -3.974222 \\
31 & 0.673865 & 13.1858 & -12.511935 \\
32 & 0.00617383 & 0.0256009 & -0.01942707 \\
33 & 0.046445 & 0.83168 & -0.785235 \\
34 & 1.95213 & 32.5976 & -30.64547 \\
35 & 4.56492 & 81.1747 & -76.60978 \\
36 & 0.0119262 & 0.103982 & -0.0920558 \\
37 & 0.00323845 & 0.0298058 & -0.02656735 \\
38 & 0.0786318 & 0.723296 & -0.6446642 \\
39 & 0.0935837 & 1.57451 & -1.4809263 \\
\end{tabular}
\end{table}

Oczywiście ta tabela jest analogiczna do poprzedniej, ale tym razem zamiast wartości Celu mamy czas działania samej części \textit{2-Opt}.\\
Jak widać, wszytskie różnice są ujemne, ale (w połączeniu z danymi otrzymanymi w tabeli 1) można powiedzieć, że łatwo uzależnić czas działania algorytmu 2-Opt od pierwotnego rozwiązania - jeżeli jest znacznie 'lepsze', to także widać, że różnice w czasie są znacznie wyższe.\\
\\
Uwwaga na boku: wydaje mi się, że lepiej będzie to przenieść do poprzedniej tabeli i zrobić p prostu zbiorcze podsumowanie

\newpage
\subsubsection{Część III}
Tabela numer 4:
\begin{table}[h!]
\centering
\begin{tabular}{c||c|c|c|c}
Nr. Instancji & 1N & FN & 1B & FB \\
1 & 3157 & 2975 & 3452 & 2998 \\
2 & 8980 & 8181 & 8980 & 8181 \\
3 & 135737 & 133953 & 129421 & 128279 \\
4 & 7579 & 7129 & 7198 & 6908 \\
5 & 8191 & 7113 & 8191 & 7113 \\
6 & 803 & 746 & 804 & 748 \\
7 & 511 & 482 & 545 & 499 \\
8 & 642 & 608 & 621 & 599 \\
9 & 27807 & 24698 & 26854 & 24815 \\
10 & 33633 & 31479 & 33612 & 31479 \\
11 & 35859 & 34543 & 35794 & 34543 \\
12 & 29158 & 25884 & 29158 & 25884 \\
13 & 34499 & 31611 & 32825 & 31611 \\
14 & 36980 & 35389 & 36980 & 35389 \\
15 & 26227 & 23660 & 26227 & 23564 \\
16 & 26947 & 24852 & 26947 & 24852 \\
17 & 27460 & 24782 & 27460 & 24782 \\
18 & 20356 & 16935 & 20356 & 16935 \\
19 & 54019 & 49201 & 54019 & 49201 \\
20 & 54019 & 49201 & 54019 & 49201 \\
21 & 68964 & 68531 & 70315 & 67864 \\
22 & 331103 & 313745 & 322807 & 310967 \\
23 & 46680 & 46680 & 49166 & 47899 \\
24 & 69297 & 67055 & 71550 & 67323 \\
25 & 120769 & 114553 & 120769 & 114553 \\
26 & 61652 & 60964 & 61652 & 60964 \\
27 & 85699 & 79564 & 85699 & 79564 \\
28 & 94683 & 92552 & 94683 & 92903 \\
29 & 58023 & 54491 & 58635 & 54604 \\
30 & 59890 & 58279 & 58151 & 58151 \\
31 & 131281 & 127230 & 131445 & 126062 \\
32 & 153462 & 130921 & 153462 & 130921 \\
33 & 2752 & 2612 & 2632 & 2606 \\
34 & 8605 & 7993 & 8700 & 7934 \\
35 & 11054 & 10540 & 10763 & 10657 \\
36 & 1554 & 1437 & 1544 & 1432 \\
37 & 830 & 796 & 801 & 751 \\
38 & 152493 & 140486 & 152493 & 140486 \\
39 & 5030 & 4578 & 4941 & 4691 \\

\end{tabular}
\end{table}

Przy czym kolejne kolumny oznaczają:
\begin{itemize}
	\item 1N - Wartość funkcji Celu dla algorytmu \textit{Nearest Neighbor} startującego z wybranego miasta
	\item FN - Wartość funkcji Celu dla algorytmu \textit{Nearest Neighbor} startującego z każdego miasta, i wybierający najlepsze
	\item 1B - Wartość funkcji Celu dla algorytmu \textit{Branching Nearest Neighbor} (z głębokością = 2) startującego z wybranego miasta
	\item FB - Wartość funkcji Celu dla algorytmu \textit{Branching Nearest Neighbor} (z głębokością = 2) startującego z każdego miasta, i wybierający najlepsze
\end{itemize}

Tabela numer 4:
\begin{table}[h!]
\centering
\begin{tabular}{c||c|c|c|c}
Nr. Instancji & 1N & FN & 1B & FB \\
1 & 0.000644468 & 0.0983119 & 0.0048885 & 1.01056 \\
2 & 2.07E-05 & 0.00103294 & 3.76E-05 & 0.00205635 \\
3 & 0.000146889 & 0.0121529 & 0.000282864 & 0.0361393 \\
4 & 0.000104299 & 0.0139546 & 0.000241166 & 0.0310817 \\
5 & 0.000134752 & 0.0224471 & 0.000239433 & 0.0382092 \\
6 & 6.75E-05 & 0.00723383 & 0.000454636 & 0.0460773 \\
7 & 2.06E-05 & 0.00110371 & 0.000153318 & 0.00747459 \\
8 & 6.48E-05 & 0.0034514 & 0.000231878 & 0.0173178 \\
9 & 6.99E-05 & 0.00736386 & 0.00012696 & 0.0131532 \\
10 & 0.000222951 & 0.0240762 & 0.00021405 & 0.0365781 \\
11 & 0.000252144 & 0.0555201 & 0.000354758 & 0.0790306 \\
12 & 9.39E-05 & 0.00853953 & 0.000109656 & 0.0112842 \\
13 & 0.000131333 & 0.0222619 & 0.000253897 & 0.0373115 \\
14 & 0.000236748 & 0.0511064 & 0.000379147 & 0.0778277 \\
15 & 6.77E-05 & 0.00713671 & 0.000131215 & 0.0138976 \\
16 & 0.000125232 & 0.00788403 & 0.000222792 & 0.0115013 \\
17 & 6.63E-05 & 0.00694331 & 0.000201731 & 0.0129614 \\
18 & 5.77E-05 & 0.00701994 & 0.000113084 & 0.0166362 \\
19 & 0.000421003 & 0.166132 & 0.000678634 & 0.260951 \\
20 & 0.000420879 & 0.155087 & 0.000638746 & 0.240323 \\
21 & 0.0102591 & 14.3142 & 0.0157081 & 24.5947 \\
22 & 0.0042375 & 4.41593 & 0.0108562 & 11.6176 \\
23 & 5.57E-05 & 0.00624229 & 0.000471479 & 0.0663509 \\
24 & 8.92E-05 & 0.00915955 & 0.000177797 & 0.0226451 \\
25 & 8.82E-05 & 0.012921 & 0.000207303 & 0.0303293 \\
26 & 9.07E-05 & 0.0139036 & 0.000168004 & 0.0304266 \\
27 & 0.000102496 & 0.0173458 & 0.000168221 & 0.0335902 \\
28 & 0.000204923 & 0.0498144 & 0.00163079 & 0.352259 \\
29 & 0.000284032 & 0.0803066 & 0.000632295 & 0.193229 \\
30 & 0.000401408 & 0.125099 & 0.00113758 & 0.380743 \\
31 & 0.000791101 & 0.383126 & 0.0016203 & 0.873932 \\
32 & 3.31E-05 & 0.00248934 & 6.17E-05 & 0.00508362 \\
33 & 0.000155953 & 0.0351781 & 0.000429553 & 0.113054 \\
34 & 0.00137766 & 0.857353 & 0.0040416 & 2.48614 \\
35 & 0.00274642 & 2.70393 & 0.00916338 & 7.70218 \\
36 & 5.00E-05 & 0.00510014 & 0.000238881 & 0.022591 \\
37 & 7.09E-05 & 0.00241989 & 0.000155393 & 0.0108239 \\
38 & 0.000206588 & 0.049608 & 0.00139074 & 0.239691 \\
39 & 0.000318213 & 0.0560495 & 0.000668361 & 0.163452 \\
\end{tabular}
\end{table}

Oczywiście ta tabela jest analogiczna do poprzedniej, ale tym razem zamiast wartości Celu mamy czas działania.\\
\\
Uwaga ogólna: tutaj warto by się w sumie chwile zastanowić, co z czym porównywać, ale początek już jakiś jest... 

\subsection{Wnioski}

\section*{Drobne uwagi}

\end{document}