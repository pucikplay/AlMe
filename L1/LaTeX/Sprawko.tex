% uklad dokumentu
	\documentclass{article}
	\usepackage{xparse}
	\usepackage[margin=1.5cm]{geometry}
    \usepackage{enumerate} 
	\frenchspacing
    \linespread{1.0}
    \setlength{\parindent}{0pt}

% jezyk polski
	\usepackage[T1]{fontenc}
	\usepackage[polish]{babel}
	\usepackage[utf8]{inputenc}
 
% pakiety matematyczne
    \let\lll\undefined
    \usepackage{mathtools}
	\usepackage{amssymb}
    \usepackage{amsthm}
	\usepackage{amsmath}
	\usepackage{amsfonts}
	\usepackage{tikz}
	
% pakiety do automatów
	\usetikzlibrary{automata, arrows.meta, positioning, arrows}

    \title{\textbf{Algorytmy Metaheurystyczne\\Komiwojażer Heurystycznie}}
    \author{Gabriel Budziński(254609)\\Franciszek Stepek (256310)}
    \date{}
    
    
\begin{document} 
\maketitle

\section*{Przedmowa}
Sprawko na metaheurystykę

\section{Podsumowanie złożoności obliczeniowych implementacji}
\subsection{K-Random}
Co nieco o nim

\subsection{Nearest Neighbor}
Co nieco o nim

\subsection{Nearest Branching Neighbor}
Co nieco o nim

\subsection{2-Opt}
Co nieco o nim

\subsection{3-Opt}
Co nieco o nim

\section{Opis eksperymentów}
\subsection{Implementacja}
C/Cpp

\subsection{Sprzęt}
16GB Ramu

\subsubsection{Pececik}
6 rdzonków
\subsubsection{Lapek}
4 rdzonki

\subsection{Instancje}
\subsubsection{Przykładziki ze stronki}

\subsubsection{Instancje losowe}

\subsection{Metodologia/cel}

\subsection{Opis wyników}

\subsection{Wnioski}

\section*{Drobne uwagi}

\end{document}