% uklad dokumentu
	\documentclass{article}
	\usepackage{xparse}
	\usepackage[margin=1.5cm]{geometry}
    \usepackage{enumerate} 
	\frenchspacing
    \linespread{1.0}
    \setlength{\parindent}{0pt}

% jezyk polski
	\usepackage[T1]{fontenc}
	\usepackage[polish]{babel}
	\usepackage[utf8]{inputenc}
 
% pakiety matematyczne
    \let\lll\undefined
    \usepackage{mathtools}
	\usepackage{amssymb}
    \usepackage{amsthm}
	\usepackage{amsmath}
	\usepackage{amsfonts}
	\usepackage{tikz}
	
% pakiety do automatów
	\usetikzlibrary{automata, arrows.meta, positioning, arrows}

    \title{\textbf{Algorytmy Metaheurystyczne\\Komiwojażer Heurystycznie}}
    \author{Gabriel Budziński(254609)\\Franciszek Stepek (256310)}
    \date{}
    
    
\begin{document} 
\maketitle

\section*{Przedmowa}
Sprawko na metaheurystykę

\section{Podsumowanie złożoności obliczeniowych implementacji}
\subsection{K-Random}
Algorytm polega na losowaniu kolejności odwiedzania wierzchołków. Każde z wylosowanych rozwiązań jest porównywane z poprzednim najlepszym. W naszej implementacji losowanie rozwiązania odbywało się w czasie liniowym. W takim samym czasie obliczana jest również funkcja celu. Dla $k$ prób złożoność jest w takim razie rzędu $O(k\cdot n\cdot n) = O(k \cdot n^2)$.

\subsection{Nearest Neighbor}
Algorytm polega na zachłannym szukaniu ścieżkirozpoczynając od zadanego wierzchołka i przechodząc do najbliższego jeszcze nie odwiedzonego. Złożoność jest rzędu $O(n^2)$, ponieważ w każdym wierzchołku musimy przeszukać zbiór pozostałych, aby znaleźć najbliższy. Chcąc uzyskać lepsze wyniki, możemy zainicjować algorytm we wszystkich wierzchołkach i wybrać najlepszą trasę. Przy takiej modyfikacji złożoność zwiększa się do $O(n^2 \cdot n + n\cdot n) = O(n^3)$, ponieważ pierwotny algorytm wykonujemy $n$ razy i $n$-krotnie musimy obliczyć wartość funkcji celu.

\subsection{Nearest Branching Neighbor}
Algorytm jest modyfikacją powyższego algorytmu, która rozstrzyga, który wierzchołek obrać jako następny w przypadku równej odległości do nich. Złożoność jest bardzo zależna od stryktury problemu i głębokości zagłębiania przeszukiwań. Zakładając, że problem ma strukturę drzewa, gdzie wierzchołek ma dwa równo odległe od siebie następniki, każdy poziom przeszukiwań ma wzrost wykładniczy.

\subsection{2-Opt}
Algorytm polega na wybieraniu kawałka istniejącej już trasy, a następnie wstawienie go z powrotem, ale w odwrotnej kolejności. Czynność tę powtarzamy, aż nie będziemy już uzyskiwać zysku długości trasy. Złożoność algorytmu można ulepszyć nie wyliczając za każdym odwróceniem funkcji celu, a jedynie odejmować i dodawać wartości dwóch usuniętych i dwóch dodanych krawędzi. Złożoność obliczeniowa w naszej implementacji jest rzędu $O(n^2\cdot n) = O(n^3)$, ponieważ operacja zmiany kolejności kawałka trasy jest $O(n)$.

\subsection{3-Opt}
Algorytm jest analogiczny do powyższego, z tą różnicą, że teraz dzielimy trasę na trzy, a nie na dwa kawałki.

\section{Opis eksperymentów}
\subsection{Implementacja}
Algorytmy implementujemy w języku \texttt{C++}, odległości między wierzchołkami są przechowywane jako pełne tablice dwuwymiarowe typu \texttt{int}, a trasy są w kontenerach \texttt{vector}, co ułatwia operacje odwracania i mieszania.

\subsection{Sprzęt}
Programy były testowane na dwóch maszynach, laptopie \textit{Lenovo} i komputerze stacjonarnym. Obie jednostki są wyposażone w procesor architektury \texttt{x86} marki \texttt{intel} oraz 16GB pamięci RAM.

\subsubsection{Pececik}
Komputer stacjonarny posiada procesor i5-10600K (o obniżonym napięciu operacyjnym).
\subsubsection{Lapek}
Laptop posiada procesor i7-niepamiętamktóry.

\subsection{Instancje}
\subsubsection{Przykłady TSPLIB}
W części eksperymentów użyto instancji euklidejskiego problemu komiwojażera.

\subsubsection{Instancje losowe}
W celu zwiększenia liczności i dokładności testów spreparowano losowo generowane instancje eukidejskiego problemu komiwojażera.

\subsection{Metodologia/cel}

\subsection{Opis wyników}
Testy Wilcoxona dla poszczególnych algorytmów:
\begin{itemize}
	\item W pierwszej dużej tabeli będziemy mieli zestawienie 3 algorytmów
	\item W drugiej (Już nieco 'ciekawszej') Zobaczymy jak na rozwiązanie \textit{2-Opt} wpływają warunki początkowe (zarówno czasowo, jak i w kontekście funkcji Celu)
	\item W trzeciej otrzymamy większe zestawienie różnych przypadków dla wariantów \textit{Nearest Neighbor}
\end{itemize}

\newpage
\subsubsection{Część I}
Tabela numer 1:
\begin{table}[h!]
\centering
\begin{tabular}{c||c|c|c|c|c c c|c|c} 
Nr. Instancji & 1 & 2 & 3 & 4 & 5 &  & 1-2 & 1-3 & 4-5 \\
\hline
1 & 3157 & 32621 & 2846 & 2922 & 33913 &  & -29464 & 311 & -30991 \\
2 & 8980 & 29990 & 8385 & 8746 & 25809 &  & -21010 & 595 & -17063 \\
3 & 135737 & 610252 & 126892 & 128097 & 622778 &  & -474515 & 8845 & -494681 \\
4 & 7579 & 47194 & 7098 & 6741 & 49975 &  & -39615 & 481 & -43234 \\
5 & 8191 & 53545 & 7157 & 7372 & 54720 &  & -45354 & 1034 & -47348 \\
6 & 803 & 3509 & 714 & 677 & 3560 &  & -2706 & 89 & -2883 \\
7 & 511 & 1586 & 454 & 443 & 1750 &  & -1075 & 57 & -1307 \\
8 & 642 & 2606 & 561 & 595 & 2645 &  & -1964 & 81 & -2050 \\
9 & 27807 & 168501 & 23737 & 23006 & 172442 &  & -140694 & 4070 & -149436 \\
10 & 33633 & 270344 & 28612 & 29756 & 242459 &  & -236711 & 5021 & -212703 \\
11 & 35859 & 331888 & 31749 & 34109 & 345599 &  & -296029 & 4110 & -311490 \\
12 & 29158 & 154024 & 25393 & 25014 & 174054 &  & -124866 & 3765 & -149040 \\
13 & 34499 & 256477 & 28435 & 29312 & 253834 &  & -221978 & 6064 & -224522 \\
14 & 36980 & 338421 & 33375 & 34698 & 339346 &  & -301441 & 3605 & -304648 \\
15 & 26227 & 174278 & 23107 & 22850 & 178774 &  & -148051 & 3120 & -155924 \\
16 & 26947 & 173366 & 24721 & 25011 & 174970 &  & -146419 & 2226 & -149959 \\
17 & 27460 & 170965 & 24459 & 23810 & 154305 &  & -143505 & 3001 & -130495 \\
18 & 20356 & 126612 & 18281 & 15972 & 121878 &  & -106256 & 2075 & -105906 \\
19 & 54019 & 593697 & 49141 & 46246 & 606262 &  & -539678 & 4878 & -560016 \\
20 & 54019 & 582949 & 49141 & 45758 & 575474 &  & -528930 & 4878 & -529716 \\
21 & 68964 & 1405305 & 60930 & 62216 & 1415877 &  & -1336341 & 8034 & -1353661 \\
22 & 331103 & 6434381 & 282758 & 279968 & 6392975 &  & -6103278 & 48345 & -6113007 \\
23 & 46680 & 597495 & 45340 & 49462 & 572145 &  & -550815 & 1340 & -522683 \\
24 & 69297 & 735352 & 61910 & 64640 & 677100 &  & -666055 & 7387 & -612460 \\
25 & 120769 & 842650 & 110103 & 107601 & 882692 &  & -721881 & 10666 & -775091 \\
26 & 61652 & 821860 & 61400 & 68116 & 823348 &  & -760208 & 252 & -755232 \\
27 & 85699 & 988886 & 79996 & 81923 & 1028352 &  & -903187 & 5703 & -946429 \\
28 & 94683 & 1638917 & 87542 & 88659 & 1635424 &  & -1544234 & 7141 & -1546765 \\
29 & 58023 & 1137733 & 56765 & 55439 & 1183487 &  & -1079710 & 1258 & -1128048 \\
30 & 59890 & 766136 & 52403 & 56468 & 791951 &  & -706246 & 7487 & -735483 \\
31 & 131281 & 1940053 & 118468 & 119774 & 1962350 &  & -1808772 & 12813 & -1842576 \\
32 & 153462 & 586378 & 121315 & 121297 & 568936 &  & -432916 & 32147 & -447639 \\
33 & 2752 & 22583 & 2497 & 2684 & 22420 &  & -19831 & 255 & -19736 \\
34 & 8605 & 110285 & 7566 & 7628 & 114431 &  & -101680 & 1039 & -106803 \\
35 & 11054 & 183204 & 9415 & 9627 & 182635 &  & -172150 & 1639 & -173008 \\
36 & 1554 & 8755 & 1418 & 1346 & 8903 &  & -7201 & 136 & -7557 \\
37 & 830 & 3585 & 770 & 775 & 3466 &  & -2755 & 60 & -2691 \\
38 & 152493 & 1629073 & 139697 & 140164 & 1576951 &  & -1476580 & 12796 & -1436787 \\
39 & 5030 & 41053 & 4418 & 4294 & 42391 &  & -36023 & 612 & -38097 \\
 \end{tabular}
\end{table}

Przy czym kolejne kolumny 'numeryczne' oznaczają:
\begin{itemize}
	\item 1 - Wartość funkcji Celu dla algorytmu \textit{Nearest Neighbor} startującego z wybranego 'miasta'
	\item 2 - Wartość funkcji Celu dla algorytmu \textit{K-Random} działającego tak długo, jak z kolumny 1
	\item 3 - Wartość funkcji Celu dla algorytmu \textit{2-Opt} startującego z rozwiązania w punkcie 1
	\item 4 - Wartość funkcji Celu dla algorytmu \textit{2-Opt} startującego z losowej instancji
	\item 5 - Wartość funkcji Celu dla algorytmu \textit{K-Random} działającego tak długo, jak z kolumny 4
\end{itemize}

A teraz jak rozumieć kolejne kolumny 'różnicowe':
\begin{itemize}
	\item 1-2 - Porównanie działania KR z NN dla tego samego budżetu obliczeniowego
	\item 1-3 - 'Dowód', że 2O rzeczywiście poprawia rozwiązanie startowe
	\item 4-5 - Porównanie działania KR z 2O dla tego samego budżetu obliczeniowego
\end{itemize}

Zauważmy, że we wszystkich tych kolumnach każda z wartości jest tego samego znaku, zatem bez dokładniejszej analizy można powiedzieć, że wartość statystyki testowej dla testu Wilcoxona będzie równa 0, zatem jednoznacznie powie nam, który z algorytmów zwraca zawsze lepsze rozwiązanie. Pojawia się nam zatem następująca zależność (gdzie znak '<' oznacza, że lewa wartość zwraca nam 'gorsze' rozwiązanie od prawego):
\[\textit{K-Random} < \textit{Nearest Neighbor} < \textit{2-Opt}\]

\newpage
\subsubsection{Część II}
Tabela numer 2:
\begin{table}[h!]
\centering
\begin{tabular}{c||c|c||c|c||c c c|c}
Nr. Instancji & NN & KR & NN-KR & |NN-KR| & Rangi & & NN(T) & KR(T)\\
\hline
1 & 2846 & 2992 & -146 & 146 & 11 &  & 0.115728 & 3.10199 \\
2 & 8385 & 7967 & 418 & 418 & 16 &  & 0.00120772 & 0.00922684 \\
3 & 126892 & 124006 & 2886 & 2886 & 29 &  & 0.0187193 & 0.145382 \\
4 & 7098 & 6988 & 110 & 110 & 8 &  & 0.0138299 & 0.184798 \\
5 & 7157 & 7080 & 77 & 77 & 6.5 &  & 0.0393003 & 0.338091 \\
6 & 714 & 689 & 25 & 25 & 4 &  & 0.00987954 & 0.0552343 \\
7 & 454 & 461 & -7 & 7 & 2 &  & 0.00119699 & 0.0103625 \\
8 & 561 & 567 & -6 & 6 & 1 &  & 0.00364293 & 0.0333502 \\
9 & 23737 & 24667 & -930 & 930 & 19 &  & 0.0154935 & 0.137051 \\
10 & 28612 & 29780 & -1168 & 1168 & 21 &  & 0.053393 & 0.313875 \\
11 & 31749 & 32634 & -885 & 885 & 18 &  & 0.144667 & 0.836886 \\
12 & 25393 & 24066 & 1327 & 1327 & 22 &  & 0.0138314 & 0.0914137 \\
13 & 28435 & 29226 & -791 & 791 & 17 &  & 0.037316 & 0.41158 \\
14 & 33375 & 33294 & 81 & 81 & 7 &  & 0.109228 & 0.997454 \\
15 & 23107 & 22870 & 237 & 237 & 13 &  & 0.0119046 & 0.102974 \\
16 & 24721 & 23678 & 1043 & 1043 & 20 &  & 0.01084 & 0.0888192 \\
17 & 24459 & 24275 & 184 & 184 & 12 &  & 0.0122842 & 0.114743 \\
18 & 18281 & 15962 & 2319 & 2319 & 26 &  & 0.0165922 & 0.0705111 \\
19 & 49141 & 46422 & 2719 & 2719 & 28 &  & 0.322236 & 4.21766 \\
20 & 49141 & 45633 & 3508 & 3508 & 32 &  & 0.343797 & 2.56742 \\
21 & 60930 & 61190 & -260 & 260 & 14 &  & 27.5015 & 656.948 \\
22 & 282758 & 287980 & -5222 & 5222 & 34 &  & 11.1163 & 179.428 \\
23 & 45340 & 52126 & -6786 & 6786 & 38 &  & 0.0065809 & 0.110337 \\
24 & 61910 & 67957 & -6047 & 6047 & 36 &  & 0.0144687 & 0.196237 \\
25 & 110103 & 103341 & 6762 & 6762 & 37 &  & 0.0155958 & 0.195165 \\
26 & 61400 & 66836 & -5436 & 5436 & 35 &  & 0.00640445 & 0.246059 \\
27 & 79996 & 82941 & -2945 & 2945 & 30 &  & 0.0186396 & 0.358017 \\
28 & 87542 & 90788 & -3246 & 3246 & 31 &  & 0.062336 & 0.966489 \\
29 & 56765 & 58313 & -1548 & 1548 & 24 &  & 0.0857857 & 2.09062 \\
30 & 52403 & 56508 & -4105 & 4105 & 33 &  & 0.221278 & 4.1955 \\
31 & 118468 & 119808 & -1340 & 1340 & 23 &  & 0.673865 & 13.1858 \\
32 & 121315 & 118891 & 2424 & 2424 & 27 &  & 0.00617383 & 0.0256009 \\
33 & 2497 & 2624 & -127 & 127 & 10 &  & 0.046445 & 0.83168 \\
34 & 7566 & 7643 & -77 & 77 & 6.5 &  & 1.95213 & 32.5976 \\
35 & 9415 & 9687 & -272 & 272 & 15 &  & 4.56492 & 81.1747 \\
36 & 1418 & 1399 & 19 & 19 & 3 &  & 0.0119262 & 0.103982 \\
37 & 770 & 799 & -29 & 29 & 5 &  & 0.00323845 & 0.0298058 \\
38 & 139697 & 137507 & 2190 & 2190 & 25 &  & 0.0786318 & 0.723296 \\
39 & 4418 & 4531 & -113 & 113 & 9 &  & 0.0935837 & 1.57451 \\
\end{tabular}
\end{table}

Przy czym kolejne kolumny oznaczają:
\begin{itemize}
	\item NN - Wartość funkcji Celu dla algorytmu \textit{2-Opt} startującego z rozwiązania znalezionego przez algorytm \textit{Nearest Neighbor}
	\item KR - Wartość funkcji Celu dla algorytmu \textit{2-Opt} startującego z rozwiązania znalezionego przez algorytm \textit{K-Random} (Startowe rozwiązanie w tym samym czasie, co startowe dla NN)
	\item Kolejne 2  kolumny znaczą dokładnie to, co ich nazwa, natomist ostatnia wyznacza rangi dla testu Wilcoxona
\end{itemize}

A teraz policzmy jeszcze tylko 2 sumy dla Wilcoxona:
\begin{itemize}
	\item $T_+ = 315.5$
	\item $T_- = 432.5$
\end{itemize}
Zatem nasza statystyka końcowa będzie wynosiła $315.5$. No jest to dosyć dużo, a fakt, że wynosi ona około 1/2 sumy obu statystyk sugeruje, że w istocie obie metody działają 'bardzo podobnie'\\

Prawa strona tabeli jest analogiczna do lewej, ale tym razem zamiast wartości Celu mamy czas działania samej części \textit{2-Opt}.\\
Jak widać, wszytskie różnice są ujemne, ale (w połączeniu z danymi otrzymanymi w tabeli 1) można powiedzieć, że łatwo uzależnić czas działania algorytmu 2-Opt od pierwotnego rozwiązania - jeżeli jest znacznie 'lepsze', to także widać, że różnice w czasie są znacznie wyższe.\\

\newpage
\subsubsection{Część III}
Tabela numer 3:
\begin{table}[h!]
\centering
\begin{tabular}{c||c|c|c|c c c|c|c|c}
Nr.In & 1N & FN & 1B & FB &  & 1N(T) & FN(T) & 1B(T) & FB(T) \\
\hline
1 & 3157 & 2975 & 3452 & 2998 &  & 0.000644468 & 0.0983119 & 0.0048885 & 1.01056 \\
2 & 8980 & 8181 & 8980 & 8181 &  & 2.07E-05 & 0.00103294 & 3.76E-05 & 0.00205635 \\
3 & 135737 & 133953 & 129421 & 128279 &  & 0.000146889 & 0.0121529 & 0.000282864 & 0.0361393 \\
4 & 7579 & 7129 & 7198 & 6908 &  & 0.000104299 & 0.0139546 & 0.000241166 & 0.0310817 \\
5 & 8191 & 7113 & 8191 & 7113 &  & 0.000134752 & 0.0224471 & 0.000239433 & 0.0382092 \\
6 & 803 & 746 & 804 & 748 &  & 6.75E-05 & 0.00723383 & 0.000454636 & 0.0460773 \\
7 & 511 & 482 & 545 & 499 &  & 2.06E-05 & 0.00110371 & 0.000153318 & 0.00747459 \\
8 & 642 & 608 & 621 & 599 &  & 6.48E-05 & 0.0034514 & 0.000231878 & 0.0173178 \\
9 & 27807 & 24698 & 26854 & 24815 &  & 6.99E-05 & 0.00736386 & 0.00012696 & 0.0131532 \\
10 & 33633 & 31479 & 33612 & 31479 &  & 0.000222951 & 0.0240762 & 0.00021405 & 0.0365781 \\
11 & 35859 & 34543 & 35794 & 34543 &  & 0.000252144 & 0.0555201 & 0.000354758 & 0.0790306 \\
12 & 29158 & 25884 & 29158 & 25884 &  & 9.39E-05 & 0.00853953 & 0.000109656 & 0.0112842 \\
13 & 34499 & 31611 & 32825 & 31611 &  & 0.000131333 & 0.0222619 & 0.000253897 & 0.0373115 \\
14 & 36980 & 35389 & 36980 & 35389 &  & 0.000236748 & 0.0511064 & 0.000379147 & 0.0778277 \\
15 & 26227 & 23660 & 26227 & 23564 &  & 6.77E-05 & 0.00713671 & 0.000131215 & 0.0138976 \\
16 & 26947 & 24852 & 26947 & 24852 &  & 0.000125232 & 0.00788403 & 0.000222792 & 0.0115013 \\
17 & 27460 & 24782 & 27460 & 24782 &  & 6.63E-05 & 0.00694331 & 0.000201731 & 0.0129614 \\
18 & 20356 & 16935 & 20356 & 16935 &  & 5.77E-05 & 0.00701994 & 0.000113084 & 0.0166362 \\
19 & 54019 & 49201 & 54019 & 49201 &  & 0.000421003 & 0.166132 & 0.000678634 & 0.260951 \\
20 & 54019 & 49201 & 54019 & 49201 &  & 0.000420879 & 0.155087 & 0.000638746 & 0.240323 \\
21 & 68964 & 68531 & 70315 & 67864 &  & 0.0102591 & 14.3142 & 0.0157081 & 24.5947 \\
22 & 331103 & 313745 & 322807 & 310967 &  & 0.0042375 & 4.41593 & 0.0108562 & 11.6176 \\
23 & 46680 & 46680 & 49166 & 47899 &  & 5.57E-05 & 0.00624229 & 0.000471479 & 0.0663509 \\
24 & 69297 & 67055 & 71550 & 67323 &  & 8.92E-05 & 0.00915955 & 0.000177797 & 0.0226451 \\
25 & 120769 & 114553 & 120769 & 114553 &  & 8.82E-05 & 0.012921 & 0.000207303 & 0.0303293 \\
26 & 61652 & 60964 & 61652 & 60964 &  & 9.07E-05 & 0.0139036 & 0.000168004 & 0.0304266 \\
27 & 85699 & 79564 & 85699 & 79564 &  & 0.000102496 & 0.0173458 & 0.000168221 & 0.0335902 \\
28 & 94683 & 92552 & 94683 & 92903 &  & 0.000204923 & 0.0498144 & 0.00163079 & 0.352259 \\
29 & 58023 & 54491 & 58635 & 54604 &  & 0.000284032 & 0.0803066 & 0.000632295 & 0.193229 \\
30 & 59890 & 58279 & 58151 & 58151 &  & 0.000401408 & 0.125099 & 0.00113758 & 0.380743 \\
31 & 131281 & 127230 & 131445 & 126062 &  & 0.000791101 & 0.383126 & 0.0016203 & 0.873932 \\
32 & 153462 & 130921 & 153462 & 130921 &  & 3.31E-05 & 0.00248934 & 6.17E-05 & 0.00508362 \\
33 & 2752 & 2612 & 2632 & 2606 &  & 0.000155953 & 0.0351781 & 0.000429553 & 0.113054 \\
34 & 8605 & 7993 & 8700 & 7934 &  & 0.00137766 & 0.857353 & 0.0040416 & 2.48614 \\
35 & 11054 & 10540 & 10763 & 10657 &  & 0.00274642 & 2.70393 & 0.00916338 & 7.70218 \\
36 & 1554 & 1437 & 1544 & 1432 &  & 5.00E-05 & 0.00510014 & 0.000238881 & 0.022591 \\
37 & 830 & 796 & 801 & 751 &  & 7.09E-05 & 0.00241989 & 0.000155393 & 0.0108239 \\
38 & 152493 & 140486 & 152493 & 140486 &  & 0.000206588 & 0.049608 & 0.00139074 & 0.239691 \\
39 & 5030 & 4578 & 4941 & 4691 &  & 0.000318213 & 0.0560495 & 0.000668361 & 0.163452 \\


\end{tabular}
\end{table}

Przy czym kolejne kolumny oznaczają:
\begin{itemize}
	\item 1N - Wartość funkcji Celu dla algorytmu \textit{Nearest Neighbor} startującego z wybranego miasta
	\item FN - Wartość funkcji Celu dla algorytmu \textit{Nearest Neighbor} startującego z każdego miasta, i wybierający najlepsze
	\item 1B - Wartość funkcji Celu dla algorytmu \textit{Branching Nearest Neighbor} (z głębokością = 2) startującego z wybranego miasta
	\item FB - Wartość funkcji Celu dla algorytmu \textit{Branching Nearest Neighbor} (z głębokością = 2) startującego z każdego miasta, i wybierający najlepsze
\end{itemize}

Zapisanie wszystkich obliczeń potrzebnych do wykkonania testu Wilcoxona zajęłoby mnóstwo miejsca, dlatego też pominiemy ich zapis, a jedynie podamy otrzymane wynika dla podanych porównań:
\begin{itemize}
	\item 1N-FN - Jak można się było spodziewać (i co także widać w tabelce), przy wykonywaniu algorytmu \textit{Nearest Neighbor} z każdego  'miasta' 'jakość' rozwiązania poprawiła się (albo najgorzej - nie zmieniła się) w każdym przypadku, dlatego wartość statystyki wynosi 0.
	\item 1B-FB - Mamy tutaj sytuację analogiczną do powyższej.
	\item 1N-FB - Tak samo jak w poprzednich widać tutaj znaczną przewagę startowania z każdego 'miasta', pomimo różnych metod.
	\item 1N-1B - Tutaj już można zaobserwować pewne różnice, pomimo tego, że w wielu przypadkach mamy takie same wartośći rozwiązania. Nasze statystyki wynoszą tutaj:
	\begin{itemize}
		\item $T_+ = 163$
		\item $T_- = 113$
	\end{itemize}
	Zatem można powiedzieć, że są na tyle bliskie siebe, by stwierdzić, że podane algorytmy względnie podobne, chociaż widać przewagę opcji \textit{Branching}.
	\item FN-FB - Analogicznie do powyższego:
	\begin{itemize}
		\item $T_+ = 142$
		\item $T_- = 111$
	\end{itemize}
	Wnioski identyczne jak powyżej.
	\item 1B-FN - Przypadek dosyć ciekawy, ponieważ dla jednej instancji (numer 30) algorytm \textit{Branching} z jednego miasta zachował się lepiej, niż algorytm standardowy startujący z każdego 'miasta'. (Jeżeli liczyć statystyki 'porządnie', to otrzymamy $8$, co można potraktować jako 'wyjątek potwierdzający regułę'.
\end{itemize}

Jeżeli chodzi zaś o czas działania, to z części z czasem ('XY(T)') widać, że pojawia się nam dosyć jednoznaczne porównanie co do prędkości (choć tu także się pojawiło 1 odstępstwo, dla instancji nr 10 \textit{Branching} okazał się szybszy od zwykłego algorytmu, choć przy tak małej różnicy w czasie, możne to 'zaniedbać'):
\[1N > 1B > FN > FB\]
Gdzie mamy od lewej do prawej jako od najszybciej działającego, do najdłuższego. Takich wyników można się było zresztą spodziewać, ponieważ wynikają one z samych konstrukcji i złożoności poszczególnych algorytmów.
\subsection{Wnioski}

\section*{Drobne uwagi}

\end{document}